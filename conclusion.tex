In this paper, we have described several alternate mechanisms to substitute for suppressions in  small-cell tabulations of business 
microdata, with the goal of improving analytic validity while maintaining a sufficiently high 
standard of disclosure limitation. Neither mechanism fundamentally changes the existing 
suppression methodology, rather, the mechanisms work to fill in the holes created by the 
suppression methodology. In particular, the first methodology (Algorithm 1) can be used 
ex-post, after initial publication of tabulations with cell suppressions. 

Leveraging the availability of a high-quality synthetic dataset
(the Synthetic LBD) with proven disclosure limitation efficiency and analytic validity \cite{KinneyEtAl2011}, the first 
method is very simple, but may suffer from seam biases and time-inconsistency. The second 
method aims to improve on that by ``blending in'' real establishments after the need for suppressions has disappeared, which may slightly 
reduce analytic validity in time periods where the strict application of the suppression 
algorithms would no longer impose any constraints, but improving on the time-series properties 
of the released data. 

%\margincomment{LV}{This section added.}
For reference, we have also used a noise-infused version of the BDS, and performs similarly if not better. 

The (preliminary) results do not bear out our hypothesis that the use of microdata for 
prolonged periods of time improves the analytic validity of the data. However, we refrain from 
definitive conclusions at this time, due to differences in the underlying microdata that 
contaminate the current results. Current improvements in the upcoming next release of both 
the existing BDS (expected in late 2015) and a new release of the Synthetic LBD will need to be 
incorporated for a more consistent analysis. Clearly, the success of our proposed methods 
depends heavily on the analytic validity of the underlying synthetic data being used.

Recent developments to improve the micro-level analytic validity of the 
\ac{SynLBD} \cite{CES-WP-2014-12} should improve the analytic validity of the mechanisms 
proposed here as well.  We also compare our proposed mechanisms to the actual published, but otherwise unmodified \ac{BDS}. Comparing  post-publication improvements to a table with suppressions \cite{HolanEtAl2010} will inevitably lead to an apparent reduction in the utility of this particular approach. Finally, the approach relies on continuous availability of synthetic microdata with analytical validity. Other approaches rely on fewer data points, and thus may be favored due to lower implementation costs.




