% $Id: intro.tex 1773 2015-12-18 22:07:18Z lv39 $ 
% $URL: https://forge.cornell.edu/svn/repos/lv39_papers/SynLBD/text/SJIAOS2015/intro.tex $

Statistics based on detailed business data are increasingly relied upon to make informed 
decisions by firms and governments. Novel statistics, for instance on business startups and firm 
dynamics \cite{BDS2}, are valuable additions to the toolbox of evidence-based policy and 
business decisions. At the same time, the sparsity and skewness of business data makes 
disclosure avoidance a challenge. Early \ac{CBP} statistics (before the advent of noise infusion as 
a disclosure avoidance measure) had between 10 and 40 percent of values suppressed.\footnote{Example taken from 2004 CBP, national by NAICS tabulations, across all size and NAICS cells.} 

In recent years, the use of fully or partially synthetic data has allowed the publication of 
increasingly detailed statistics. Going back to the seminal contributions of \cite{rubin93} and 
\cite{little93}, the release of statistics based on partially synthetic data in the U.S. Census 
Bureau's \ac{LODES} \cite{Ashwin2008} and \ac{ACS} 
\cite{Rodriguez2007,hawala2008,zayatzjsm2009}⁠ strikes a 
new 
balance between detailed statistics and appropriate disclosure avoidance. Other cases of using 
synthetic data for the purpose of tabulation exist
\cite{AbowdEtAl2012,RePEc:cen:wpaper:13-19}.
  Furthermore, partially 
synthetic microdata \cite{Drechsler2012,KinneyEtAl2011,ssafinal} has been released to 
end-users as a access and analysis mechanism \cite{CES-WP-2014-10}.

In this paper, we explore the use of tabulations based on partially synthetic data as a disclosure 
avoidance mechanism for certain at-risk tabulation cells. This is similar in spirit to the originally 
proposed uses of synthetic data \cite{little93,rubin93}, and follows similar uses of  
partially synthetic data in the \ac{ACS} \cite{hawala2008,zayatzjsm2009}⁠. Our approach differs in 
that we address longitudinal consistency of the data explicitly, an important feature of the 
statistics underlying our paper. 

To illustrate and implement the proposed mechanism, we use the \acf{BDS}. The \ac{BDS} were 
first released in 
2008, providing novel statistics on business startups on a comprehensive basis for the U.S. 
economy \cite{BDS2}. They have been used in a number of recent publications, addressing %\margincomment{Re MHS}{Addressed Martha's comment here.}
questions of  job creation and destruction, establishment births and deaths, and firm startups and shutdowns %\margincomment{LV}{Added additional references}
\cite{NBERw16300,RePEc:bin:bpeajo:v:43:y:2011:i:2011-02:p:73-142,10.1257/jep.28.3.3,pugsley2014grown}. 
The \ac{BDS} are sourced from confidential microdata in the \acf{LBD}. It
provides measures of business openings and closings, and job creation and destruction, by a variety of cross-classifications (firm and establishment age and size, industrial sector, and geography).
Since the first release, additional cross-tabulations have been added each year: initially provided 
only based on  firm charateristics, tabulations based on establishment characteristics were later 
added, as were additional geography cross-tabulations (Metropolitan Statistical Area, and 
Metro/Non-Metro).  Sensitive data are currently protected through suppression. However, as 
additional tabulations are being developed, at ever more detailed geographic levels, the number 
of suppressions increases dramatically.\footnote{The next set of expansions include plans to 
provide additional industry and geography detail.}  

We leverage the existence of a sophisticated partially synthetic data file the Synthetic LBD 
\cite{SynLBD20,KinneyEtAl2011}, 
henceforth \acs{SynLBD} -- in combination with the techniques first expressed in 
\cite{Gittings2009thesis} and \cite{RePEc:bes:jnlasa:v:105:i:492:y:2010:p:1347-1357} to replace 
sensitive cells with tabulations based on synthetic 
data.  A previous paper \cite{psd2014a} described early results from the implementation of the 
simplest algorithm described here. In this version, we refine those algorithms, and present new 
results. 
We start by describing the extent of suppressions in the \ac{BDS}, then lay out the algorithm to 
combine synthetic and confidential data for the purposes of tabulation. Preliminary results are 
discussed, and an outlook given on the next steps necessary to achieve a robust public-use 
tabulation. 

